\documentclass{article}
\usepackage{amsmath, amsfonts}
\begin{document}
Prove that $(A - B) \subseteq B \iff (A - B) = \emptyset$.

Assume $(A-B) \neq \emptyset$. First we consider the forward direction:

By definition 2.1.4, $(A-B)=\{x\in A \mid x\not\in B\}$, which means $(A-B)$ consists of elements in $A$ but not in $B$.

Since $(A-B) \subseteq B$, every element $x \in (A-B)$ must also belong to $B$. 
This leads to a contradiction.

Thus, there can't be any element in $(A-B)$, which implies that $(A-B) = \emptyset$.

Then we consider the reverse direction:

Since $(A-B) = \emptyset$, there can't be any element in $(A-B)$. 
By definition, an empty set is a subset of any set.

Therefore, $(A-B) \subseteq B$.

We conclude that: \[(A - B) \subseteq B \iff (A - B) = \emptyset\]
\end{document}