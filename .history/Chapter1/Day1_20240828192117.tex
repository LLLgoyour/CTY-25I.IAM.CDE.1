\documentclass{article}

\begin{document}

\title{Mathematical Assertions}
\author{Xiangcheng Dai}
\date{\today}

\maketitle
% Here are the 16 statements.

1. The points $(-1,1)$, $(2,-1)$, and $(3,0)$ lie on a line.\newline
2. If $x$ is an integer then $x^2 \geq x$.\newline
3. If $x$ is an integer then $x^3 \geq x$.\newline
4. For all real numbers $x$, $x^3 = x$.\newline
5. There exists a real number $x$ such that $x^3 = x$.\newline
6. $2 - \sqrt{2}$ is an irrational number.\newline
7. For integers $x$, $y$, and $z$, if $x+y$ is odd and $y+z$ is odd, then $x+z$ is odd.\newline
8. If $x$ is an even integer, then $x^2$ is an even integer.\newline
9. Every positive integer is the sum of distinct powers of two.\newline
10. Every positive integer is the sum of distinct powers of three.\newline
11. If $x$ is an integer then $x$ is even or $x$ is odd.\newline
12. If $x$ is an integer then $x$ cannot be both even and odd.\newline
13. Every even integer greater than 2 can be expressed as the sum of two prime numbers.\newline
14. There are infinitely many prime numbers.\newline
15. For any positive real number $x$, there exists a positive real number $y$ such that $y^2 = x$.\newline
16. Given three distinct points in space, there is one and only one plane passing through them.\newline

\section{True Statements}

\begin{enumerate}
    \item Statement 1: If $x$ is an integer then $x^2 \geq x$.\newline
          \begin{itemize}
              \item Case 1: $x = 0$.
                    \[0^2 = 0 \geq 0\]
                    So, the statement is true when $x = 0$.
              \item Case 2: $x > 0$.
                    \[x^2 > x\]
                    So, the statement is true when $x > 0$.
              \item Case 3: $x < 0$.
                    \[x^2 > x\]
                    So, the statement is true when $x < 0$.
          \end{itemize}
    \item Statement 2: Justification (if conclusive)
\end{enumerate}

\section{Uncertain Statements}

\begin{enumerate}
    \item Statement 1: Justification (if inconclusive)
    \item Statement 2: Justification (if inconclusive)
\end{enumerate}

\section{False Statements}

\begin{enumerate}
    \item The points $(-1,1)$, $(2,-1)$, and $(3,0)$ lie on a line.\newline
          We can determine if the points lie on the same line by comparing the slopes between each pair of points.
          If at least one pair of slopes are not equal, the points are not collinear.

          \begin{itemize}
              \item Slope between \((-1, 1)\) and \((2, -1)\):
                    \[
                        m_1 = \frac{-1 - 1}{2 - (-1)} = -\frac{2}{3}
                    \]

              \item Slope between \((2, -1)\) and \((3, 0)\):
                    \[
                        m_2 = \frac{0 - (-1)}{3 - 2} = 1
                    \]

              \item Slope between \((-1, 1)\) and \((3, 0)\):
                    \[
                        m_3 = \frac{0 - 1}{3 - (-1)} = \frac{-1}{4}
                    \]
          \end{itemize}

          Since \(m_1 = -\frac{2}{3}\), \(m_2 = 1\), and \(m_3 = -\frac{1}{4}\), and these slopes are not equal, the points do not lie on the same line. (I am confident that the justification I gave is conclusive)
    \item Statement 2: Justification (if conclusive)
\end{enumerate}

\section{Undecided Statements}

\begin{enumerate}
    \item Statement 1
    \item Statement 2
\end{enumerate}

\end{document}