\documentclass{article}
\usepackage{amsmath}

\begin{document}

\title{Mathematical Assertions}
\author{Xiangcheng Dai}
\date{\today}

\maketitle
% Here are the 16 statements.

1. The points $(-1,1)$, $(2,-1)$, and $(3,0)$ lie on a line.\newline
2. If $x$ is an integer then $x^2 \geq x$.\newline
3. If $x$ is an integer then $x^3 \geq x$.\newline
4. For all real numbers $x$, $x^3 = x$.\newline
5. There exists a real number $x$ such that $x^3 = x$.\newline
6. $2 - \sqrt{2}$ is an irrational number.\newline
7. For integers $x$, $y$, and $z$, if $x+y$ is odd and $y+z$ is odd, then $x+z$ is odd.\newline
8. If $x$ is an even integer, then $x^2$ is an even integer.\newline
9. Every positive integer is the sum of distinct powers of two.\newline
10. Every positive integer is the sum of distinct powers of three.\newline
11. If $x$ is an integer then $x$ is even or $x$ is odd.\newline
12. If $x$ is an integer then $x$ cannot be both even and odd.\newline
13. Every even integer greater than 2 can be expressed as the sum of two prime numbers.\newline
14. There are infinitely many prime numbers.\newline
15. For any positive real number $x$, there exists a positive real number $y$ such that $y^2 = x$.\newline
16. Given three distinct points in space, there is one and only one plane passing through them.\newline

\section{True Statements}

\begin{enumerate}
    \item Statement 2: If $x$ is an integer then $x^2 \geq x$.
          \begin{itemize}
              \item Case 1: $x = 0$.
                    \[
                        0^2 = 0
                    \]
                    So, the statement is true when $x = 0$.
              \item Case 2: $x = 1$.
                    \[
                        x^2 = x = 1
                    \]
                    So, the statement is true when $x = 1$.
              \item Case 3: $x > 1$.
                    \[
                        x^2 = x \times x > x \quad \text{since} \quad x \times x > x \quad \text{for} \quad x > 1
                    \]
                    So, the statement is true when $x > 1$.
              \item Case 4: $x < 0$.\newline
                    Let \(x\) be a negative integer, so \(x = -a\) where \(a > 0\). Then:
                    \[
                        x^2 = {(-a)}^2 = a^2 \quad \text{and} \quad x = -a
                    \]
                    Since \(a^2 > a\) for \(a > 0\), we have:
                    \[
                        x^2 = a^2 > a > -a = x
                    \]
                    So, the statement is true when $x < 0$.

                    Therefore, for all integers \(x\), \(x^2 \geq x\) is true.\newline
                    (I am confident that the justification I gave is conclusive)
          \end{itemize}
    \item Statement 5: There exists a real number $x$ such that $x^3 = x$.\newline
          Let \(x = 0\), then:
          \[
              x^3 = 0^3 = 0
          \]
          So, the statement is true when \(x = 0\).\newline
          Therefore, there exists a real number \(x\) such that \(x^3 = x\).\newline
          (I am confident that the justification I gave is conclusive)
    \item Statement 6: $2 - \sqrt{2}$ is an irrational number.\newline
          Let \(x = 2 - \sqrt{2}\). Assume that \(x = 2 - \sqrt{2}\) is a rational number, then:
          \[
              x^2 = {{(2 - \sqrt{2})}^2} = 6 - 4\sqrt{2} \Rightarrow \sqrt{2} = -\frac{x^2-6}{4}
          \]
          Since \(x\) is a rational number, \(x^2\) is a rational number, and \(-\frac{x^2-6}{4}\) is a rational number.\newline
          Since \(\sqrt{2}\) is irrational, \(-\frac{x^2-6}{4}\) is irrational. However, \(-\frac{x^2-6}{4}\) is known to be rational,
          leading to a contradiction.\newline
          Therefore, \(2 - \sqrt{2}\) is an irrational number.\newline
          (I am confident that the justification I gave is conclusive)
    \item Statement 8: If $x$ is an even integer, then $x^2$ is an even integer.\newline
          Let \(x = 2a\) where \(a\) is an integer. Then:
          \[
              x^2 = (2a)^2 = 4a^2 = 2(2a^2)
          \]
          Since \(2a^2\) is an integer, \(x^2\) is an even integer.\newline
          Therefore, if \(x\) is an even integer, then \(x^2\) is an even integer.\newline
          (I am confident that the justification I gave is conclusive)
    \item Statement 9: Every positive integer is the sum of distinct powers of two.\newline
          Let \(n\) be a positive integer. We can express \(n\) as the sum of distinct powers of two by using the binary representation of \(n\).
          For example, the binary representation of 11 is 1011, which can be expressed as \(2^3 + 2^1 + 2^0\).\newline
          Therefore, every positive integer is the sum of distinct powers of two.\newline
          (I am confident that the justification I gave is conclusive)
    \item Statement 10: Every positive integer is the sum of distinct powers of three.\newline
\end{enumerate}

\section{Uncertain Statements}

\begin{enumerate}
    \item
\end{enumerate}

\section{False Statements}

\begin{enumerate}
    \item Statement 1: The points $(-1,1)$, $(2,-1)$, and $(3,0)$ lie on a line.\newline
          We can determine if the points lie on the same line by comparing the slopes between each pair of points.
          If at least one pair of slopes are not equal, the points are not collinear.

          \begin{itemize}
              \item Slope between \((-1, 1)\) and \((2, -1)\):
                    \[
                        m_1 = \frac{-1 - 1}{2 - (-1)} = -\frac{2}{3}
                    \]

              \item Slope between \((2, -1)\) and \((3, 0)\):
                    \[
                        m_2 = \frac{0 - (-1)}{3 - 2} = 1
                    \]

              \item Slope between \((-1, 1)\) and \((3, 0)\):
                    \[
                        m_3 = \frac{0 - 1}{3 - (-1)} = -\frac{1}{4}
                    \]
          \end{itemize}

          Since \(m_1 = -\frac{2}{3}\), \(m_2 = 1\), and \(m_3 = -\frac{1}{4}\), and these slopes are not equal, the points do not lie on the same line. (I am confident that the justification I gave is conclusive)
    \item Statement 3: If $x$ is an integer then $x^3 \geq x$.
          \begin{itemize}
              \item Case 1: $x = 0$.
                    \[
                        0^3 = 0
                    \]
                    So, the statement is true when $x=0$.
              \item Case 2: $x = 1$.
                    \[
                        x^3 = x = 1
                    \]
                    So, the statement is true when $x=1$.
              \item Case 3: $x > 1$.
                    \[
                        x^3 > x
                    \]
                    So, the statement is true when $x > 1$.
              \item Case 4: $x = -1$.
                    \[
                        x^3 = x = -1
                    \]
                    So, the statement is true when $x = -1$.
              \item Case 5: $x < -1$.\newline
                    Let \(x\) be a negative integer, so \(x = -a\) where \(a > 0\).
                    Then:
                    \[
                        x^3 = {(-a)}^3 = -a^3 \quad \text{and} \quad x = -a
                    \]
                    Since \(a^3 > a\) for \(a > 0\), we have:
                    \[
                        x^3 = -a^3 < -a = x
                    \]
                    So, the statement is false when $x < -1$.\newline
                    Therefore, for integers \(x < -1\), \(x^3 \geq x\) is false.\newline
                    (I am confident that the justification I gave is conclusive)
          \end{itemize}
    \item Statement 4: For all real numbers $x$, $x^3 = x$.
          \begin{itemize}
              \item Case 1: $x = 0$.
                    \[
                        0^3 = 0
                    \]
                    So, the statement is true when $x = 0$.
              \item Case 2: $x = 1$.
                    \[
                        x^3 = x = 1
                    \]
                    So, the statement is true when $x = 1$.
              \item Case 3: $x > 1$.
                    \[
                        x^3 > x
                    \]
                    So, the statement is false when $x > 1$.
              \item Case 4: $x = -1$.
                    \[
                        x^3 = x = -1
                    \]
                    So, the statement is true when $x = -1$.
              \item Case 5: $x < -1$.
                    Let \(x\) be a negative integer, so \(x = -a\) where \(a > 0\).
                    Then:
                    \[
                        x^3 = {(-a)}^3 = -a^3 \quad \text{and} \quad x = -a
                    \]
                    Since \(a^3 > a\) for \(a > 0\), we have:
                    \[
                        x^3 = -a^3 < -a = x
                    \]
                    So, the statement is false when $x < -1$.\newline
                    Therefore, for integers \(x < -1\) and \(x > 1\), \(x^3 = x\) is false.\newline
                    (I am confident that the justification I gave is conclusive)
          \end{itemize}
    \item Statement 7: For integers $x$, $y$, and $z$, if $x+y$ is odd and $y+z$ is odd, then $x+z$ is odd.\newline
          Let \(m = x + y\), \(n = y + z\). Since \(x + y\) is odd and \(y + z\) is odd, \(m\) and \(n\) are odd integers.\newline
          Since \(m\) and \(n\) are odd, they can be expressed as \(m = 2a + 1\) and \(n = 2b + 1\) where \(a\) and \(b\) are integers.\newline
          Then:
          \[
              x + z = (x + y) + (y + z) - 2y = m + n - 2y = (2a + 1) + (2b + 1) - 2y = 2(a + b - y + 1)
          \]
          Since \(x + z\) has a factor of 2, it is even.\newline
          Therefore, the statement is false.\newline
          (I am confident that the justification I gave is conclusive)
\end{enumerate}

\section{Undecided Statements}

\begin{enumerate}
    \item Statement 1
    \item Statement 2
\end{enumerate}

\end{document}