\documentclass{article}

\begin{document}

\title{Mathematical Assertions}
\author{Xiangcheng Dai}
\date{\today}

\maketitle
% Here are the 16 statements.

1. The points $(-1,1)$, $(2,-1)$, and $(3,0)$ lie on a line.
2. If $x$ is an integer then $x^2 \geq x$.
3. If $x$ is an integer then $x^3 \geq x$.
4. For all real numbers $x$, $x^3 = x$.
5. There exists a real number $x$ such that $x^3 = x$.
6. $2 - \sqrt{2}$ is an irrational number.
7. For integers $x$, $y$, and $z$, if $x+y$ is odd and $y+z$ is odd, then $x+z$ is odd.
8. If $x$ is an even integer, then $x^2$ is an even integer.
9. Every positive integer is the sum of distinct powers of two.
10. Every positive integer is the sum of distinct powers of three.
11. If $x$ is an integer then $x$ is even or $x$ is odd.
12. If $x$ is an integer then $x$ cannot be both even and odd.
13. Every even integer greater than 2 can be expressed as the sum of two prime numbers.
14. There are infinitely many prime numbers.
15. For any positive real number $x$, there exists a positive real number $y$ such that $y^2 = x$.
16. Given three distinct points in space, there is one and only one plane passing through them.

\section{True Statements}

\begin{enumerate}
    \item Statement 1: Justification (if conclusive)
    \item Statement 2: Justification (if conclusive)
\end{enumerate}

\section{Uncertain Statements}

\begin{enumerate}
    \item Statement 1: Justification (if inconclusive)
    \item Statement 2: Justification (if inconclusive)
\end{enumerate}

\section{False Statements}

\begin{enumerate}
    \item The points $(-1,1)$, $(2,-1)$, and $(3,0)$ lie on a line:
          \textbf{Problem:} Prove whether the points \((-1, 1)\), \((2, -1)\), and \((3, 0)\) lie on the same line.

          \textbf{Solution:}

          We can determine if the points lie on the same line by comparing the slopes between each pair of points. If all the slopes are equal, the points are collinear.

          \textbf{Step 1: Calculate the slope between each pair of points.}

          \begin{itemize}
              \item Slope between \((-1, 1)\) and \((2, -1)\):
                    \[
                        m_1 = \frac{-1 - 1}{2 - (-1)} = \frac{-2}{3} = -\frac{2}{3}
                    \]

              \item Slope between \((2, -1)\) and \((3, 0)\):
                    \[
                        m_2 = \frac{0 - (-1)}{3 - 2} = \frac{1}{1} = 1
                    \]

              \item Slope between \((-1, 1)\) and \((3, 0)\):
                    \[
                        m_3 = \frac{0 - 1}{3 - (-1)} = \frac{-1}{4}
                    \]
          \end{itemize}

          \textbf{Step 2: Compare the slopes.}

          Since \(m_1 = -\frac{2}{3}\), \(m_2 = 1\), and \(m_3 = -\frac{1}{4}\), and these slopes are not equal, the points do \textbf{not} lie on the same line.

          \textbf{Conclusion:} The points \((-1, 1)\), \((2, -1)\), and \((3, 0)\) are not collinear.  (if conclusive)
    \item Statement 2: Justification (if conclusive)
\end{enumerate}

\section{Undecided Statements}

\begin{enumerate}
    \item Statement 1
    \item Statement 2
\end{enumerate}

\end{document}