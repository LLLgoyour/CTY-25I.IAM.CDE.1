\documentclass{article}
\begin{document}

Throughout the course, you may encounter the sets defined below. You are welcome to use these as appropriate.

\[
    \mathbb{N} = \text{{the set of natural numbers }} \{1,2,3,\ldots\}
\]

\[
    \mathbb{Z} = \text{{the set of integers }} \{\ldots,-3,-2,-1,0,1,2,3,\ldots\}
\]

\[
    \mathbb{Z}^+ = \text{{the set of positive integers }} \{1,2,3,\ldots\} \quad (\text{{note: }} \mathbb{Z}^+ = \mathbb{N})
\]

\[
    \mathbb{Z}^- = \text{{the set of negative integers }} \{\ldots,-3,-2,-1\}
\]

\[
    \mathbb{Z} \geq n = \text{{the set of integers greater than or equal to }} n \text{{: }} \{n,n+1,n+2,\ldots\}
\]

\[
    \mathbb{E} = \text{{the set of even integers }} \{\ldots,-4,-2,0,2,4,\ldots\}
\]

\[
    \mathbb{O} = \text{{the set of odd integers }} \{\ldots,-3,-1,1,3,\ldots\}
\]

\[
    \mathbb{Q} = \text{{the set of rational numbers }} = \left\{\frac{a}{b} \mid a,b \in \mathbb{Z}, b \neq 0\right\}
\]

\[
    \mathbb{Q}^+ = \text{{the set of positive rational numbers}}
\]

\[
    \mathbb{Q}^- = \text{{the set of negative rational numbers}}
\]

\[
    \mathbb{I} = \text{{the set of irrational numbers}}
\]

\[
    \mathbb{R} = \text{{the set of real numbers}}
\]

\[
    \mathbb{R}^+ = \text{{the set of positive real numbers}}
\]

\[
    \mathbb{R}^- = \text{{the set of negative real numbers}} \quad (\text{{note that this is not defined in the text but we define it here to be analogous to }} \mathbb{Z}^- \text{{ and }} \mathbb{Q}^-)
\]

\[
    \mathbb{R}^* = \text{{the set of nonzero real numbers}} \quad (\text{{this is defined on p. 93 and not used often in the course}})
\]

\[
    \mathbb{C} = \text{{the set of complex numbers}}
\]

\end{document}
