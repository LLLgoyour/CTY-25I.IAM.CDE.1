\documentclass{article}
\usepackage{amsmath, amsfonts, amsthm}
\begin{document}

Directions: Let $f:A\to B$ be a bijection and $A_1$ be a subset of $A$.

(a) Prove that $f(A - A_1) = B - f(A_1)$.
\begin{proof}[Proof of (a)]
    By definition, $f(A - A_1) = \{f(a)  |a\in A - A_1\}$. Since $A-A_1 =\{a\in A | a\notin A_1\}$.
    Injectivity ensures that $f(a)$ is unique for each $a\in A-A_1$, so no values in $f(A - A_1)$ overlap with those in $f(A_1)$.
    Surjectivity ensures that $f(A)=B$, meaning every $b\in B$ corresponds to some $a\in A$.
    Since $A = (A_1\cup(A-A_1))$ and $A_1\cap(A-A_1) = \emptyset$, $f(A_1)$ and $f(A-A_1)$ are disjoint, and their union equals $f(A) = B$.
    Thus, $B-f(A_1) = f(A) - f(A_1) = f(A-A_1)\Rightarrow f(A-A_1)=B-f(A_1)$.
\end{proof}
(b) Could you have proved the same result under the assumption that $f$ is surjective? Explain.

No, the result cannot be proved under the assumption that $f$ is only surjective. 
Surjectivity guranatees that $f(A) = B$, so every $b\in B$ has at least one $a\in A$ such that $f(a) = b$.
However, without injectivity, $f$ may map multiple elements of $A$ to the same element of $B$.
This would make $f(A-A_1)$ ambiguous because removing $A_1$ from $A$ may not guarantee unique elements in B.
Therefore, the equality $f(A-A_1) = B - f(A_1)$ has to also rely on $f$ being injective.

(c) Could you have proved the same result under the assumption that $f$ is injective? Explain.

No, the result cannot be proved under the assumption that $f$ is only injective. 
Injectivity ensures that $f$ maps distinct elements of $A$ to distinct elements of $B$, so $f(A-A_1)$ and $f(A_1)$ are disjoint.
Without surjectivity, $f(A)\subset B$, meaning there may exist elements in $B$ that are not mapped by $f$.
In this case, the complement $B-f(A_1)$ may include elements of $B$ that do not correspond to any element of $A$.
Therefore, the equality $f(A-A_1) = B - f(A_1)$ has to also rely on $f$ being surjective.
\end{document}