\documentclass{article}
\usepackage{amsmath, amsfonts, amsthm}
\begin{document}
Directions: Let $f:A\to B$ be a function, $A_1$ and $A_2$ be subsets of $A$, and $B_1$ and $B_2$ be subsets of $B$.

\begin{enumerate}
    \item Prove that $f^{-1}(B_1\cap B_2)=f^{-1}(B_1)\cap f^{-1}(B_2)$
    \begin{proof}
        Since $f$ is a function, it is injective (one-to-one) and surjective (onto).
    \end{proof}
    \item Prove that $f(A_1\cap A_2)\subseteq f(A_1)\cap f(A_2)$ but $f(A_1)\cap f(A_2)\subseteq f(A_1\cap A_2)$ doesn't necessarily hold.
    \begin{proof}
        Let $f:A\to B$ be a function such that $f(A_1)=B_1$ and $f(A_2)=B_2$.
    \end{proof}
    \item Explain what about the definition of functions enables the statement in (a) to be true, but precludes $f(A_1)\cap f(A_2)\subseteq f(A_1\cap A_2)$. (Hint: Think about what all counterexamples to this statement have in common.)
    \begin{proof}
        Since $f$ is a function, it is injective (one-to-one) and surjective (onto).
    \end{proof}
\end{enumerate}
\end{document}